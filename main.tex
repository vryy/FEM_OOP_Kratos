% OOFEM lecture note
\documentclass[12pt,a4paper,twoside]{article}
\usepackage[font=it, belowskip=9pt]{caption}
\usepackage{amsmath}
\usepackage{amssymb}
\usepackage{a4wide}
\usepackage{longtable}
\usepackage[english]{babel}
\usepackage{listings}
\usepackage{color}
\definecolor{hgrau}{gray}{0.8}
\usepackage{graphics}
\usepackage{framed}
\usepackage[xcolor,pdftex,leftbars]{changebar}
\usepackage{xspace}
\cbcolor{gray}
\usepackage{graphicx}
\usepackage{latexsym}
\usepackage{natbib}   % natbib citation package; \citet, \citep
\usepackage[toc,page]{appendix} %for appendix
%\usepackage{sty/vo}
% \usepackage{sty/hurga_diss}
%\usepackage{sty/nassi}
%\usepackage{sty/struktex}
\usepackage{hyperref}
\usepackage{cleveref}
\usepackage[T1]{fontenc}
\usepackage{pgf}
\usepackage{tikz}
%\usepackage{tikz-uml}

\nonstopmode

\title{\textbf{OKFEM} LECTURE NOTE}
\author{Hoang-Giang Bui \and Bac-Nam Vu}
\date{\today}

\begin{document} 

%%%%%%%%%%%%%%%%%%%%%%%%%%%%%%%%%%%%%%%%%%%%%%%%%%%%%%%%%%%%%%%%%%%%%%%%%%%%%%%
% DEFINE/REDEFINE COMMANDS
%%%%%%%%%%%%%%%%%%%%%%%%%%%%%%%%%%%%%%%%%%%%%%%%%%%%%%%%%%%%%%%%%%%%%%%%%%%%%%%
\newcommand{\xs}[1]{\raisebox{1.5ex}[-1.5ex]{#1}}
     \textwidth 165mm
     \textheight 235mm
     \topmargin -18mm
     %\textheight23.80cm
      \headsep0.8cm
%    \oddsidemargin 0mm
%    \evensidemargin \oddsidemargin
%    \columnsep7mm

\newcommand{\kratos}{\texttt{KRATOS}\xspace}
\newcommand{\python}{\texttt{Python}\xspace}
\newcommand{\git}{\texttt{git}\xspace}
\newcommand{\github}{\texttt{GitHub}\xspace}
\newcommand{\cmake}{\texttt{cmake}\xspace}
\newcommand{\boost}{\texttt{boost}\xspace}

%%%%%%%%%%%%%%%%%%%%%%%%%%%%%%%%%%%%%%%%%%%%%%%%%%%%%%%%%%%%%%%%%%%%%%%%%%%%%%%
% END OF COMMANDS
%%%%%%%%%%%%%%%%%%%%%%%%%%%%%%%%%%%%%%%%%%%%%%%%%%%%%%%%%%%%%%%%%%%%%%%%%%%%%%%

\maketitle

\section{Preparation}

\subsection{\kratos compilation}

\kratos can be compiled and installed using the handy utility script \texttt{kratos\_prepare\_system.sh}. A typical content of such a script is below
%
\begin{lstlisting}[language=bash, basicstyle=\footnotesize, frame=single]
#!/bin/sh
FILE="/tmp/out.$$"
GREP="/bin/grep"
#....
if [ "$(id -u)" != "0" ]; then
   echo "You run this script as non-root user. \
   	Please provide password to install additional libraries."
   sudo apt-get install wget git python-dev libblas-dev liblapack-dev \
   	gfortran gcc g++ libomp-dev libpthread-stubs0-dev make libtool
else
   echo "You run this script as root user."
   apt-get install wget git python-dev libblas-dev liblapack-dev gfortran \
   	gcc g++ libomp-dev libpthread-stubs0-dev make libtool
fi
echo "################################################"
echo "Clone the kratos repository."
git clone https://github.com/vryy/kratos_bcn2.git
cd kratos_bcn2
KRATOS_ROOT_PATH=$(pwd)
echo "################################################"
echo "Clone the structural_application."
cd applications
git clone https://github.com/vryy/structural_application.git
echo "################################################"
echo "Clone the ExternalSolversApplication"
git clone https://github.com/vryy/ExternalSolversApplication.git
echo "################################################"
cd ..
cd external_libraries
echo "Get and compile Cmake."
wget http://www.cmake.org/files/v2.8/cmake-2.8.12.tar.gz
tar -xvf cmake-2.8.12.tar.gz
mv cmake-2.8.12 cmake-2.8.12_src
cd cmake-2.8.12_src
./configure --prefix=$(pwd)/../cmake-2.8.12
make install -j$(nproc)
cd ..
#rm -rf cmake-2.8.12_src cmake-2.8.12.tar.gz
echo "Compile Cmake completed."
echo "################################################"
echo "Get and compile boost."
wget http://sourceforge.net/projects/boost/files/boost/1.77.0/\
	boost_1_77_0.tar.bz2/download
mv download boost_1_77_0.tar.bz2
tar -xvf boost_1_77_0.tar.bz2
mv boost_1_77_0 boost_1_77_0_src
cd boost_1_77_0_src
./bootstrap.sh --prefix=$(pwd)/../boost_1_77_0
./b2 install -j$(nproc)
cd ..
#rm -rf boost_1_77_0_src boost_1_77_0.tar.bz2
echo "Compile boost completed."
echo "################################################"
echo "Get and compile Openmpi."
wget https://download.open-mpi.org/release/open-mpi/v3.1/\
	openmpi-3.1.2.tar.bz2
tar -xvf openmpi-3.1.2.tar.bz2
mv openmpi-3.1.2 openmpi-3.1.2_src
cd openmpi-3.1.2_src
./configure --prefix=$(pwd)/../openmpi-3.1.2 --enable-mpi-thread-multiple
make install -j$(nproc)
cd ..
MPI_LOC=$(pwd)/openmpi-3.1.2
echo "Compile Openmpi completed."
echo "################################################"
cmakekratos=$(pwd)/cmake-2.8.12/bin/cmake
safe_cmakekratos=$(printf '%s\n' "$cmakekratos" | sed 's/[[\.*^$/]/\\&/g')
BOOST_LOC=$(pwd)/boost_1_77_0
export BOOST_ROOT=$BOOST_LOC
ln -s $BOOST_ROOT/lib/libboost_python27.so $BOOST_ROOT/lib/libboost_python.so
echo "Setup the environment variables."
echo "alias cmakekratos=$cmakekratos" >> ~/.bashrc
echo "export BOOST_ROOT=$BOOST_LOC" >> ~/.bashrc
echo "export LD_LIBRARY_PATH=\$BOOST_ROOT/lib:\$LD_LIBRARY_PATH" >> ~/.bashrc
echo "export MPI_ROOT=$MPI_LOC" >> ~/.bashrc
echo "export LD_LIBRARY_PATH=\$MPI_ROOT/lib:\$LD_LIBRARY_PATH" >> ~/.bashrc
echo "export KRATOS_ROOT_PATH=$KRATOS_ROOT_PATH" >> ~/.bashrc
echo "export PYTHONPATH=\$KRATOS_ROOT_PATH/libs:\$PYTHONPATH" >> ~/.bashrc
echo "export PYTHONPATH=\$KRATOS_ROOT_PATH:\$PYTHONPATH" \
	>> ~/.bashrc
echo "export LD_LIBRARY_PATH=\$KRATOS_ROOT_PATH/libs:\$LD_LIBRARY_PATH" \
	>> ~/.bashrc
echo "Setup the environment variables. Now it's compulsory to exit \
		the current bash. Exiting..."
exit
bash
cd $KRATOS_ROOT_PATH
mkdir cmake_build
cp cmake_build_templates/configure_gcc.sh cmake_build/configure.sh
echo 'Preparation completed. Please open the new terminal to compile Kratos.'
\end{lstlisting}

\subsubsection{Explanation of the installation script}

\begin{lstlisting}[language=bash, basicstyle=\footnotesize, frame=single]
#!/bin/sh
FILE="/tmp/out.$$"
GREP="/bin/grep"
#....
if [ "$(id -u)" != "0" ]; then
   echo "You run this script as non-root user. \
   	Please provide password to install additional libraries."
   sudo apt-get install wget git python-dev libblas-dev liblapack-dev \
   	gfortran gcc g++ libomp-dev libpthread-stubs0-dev make libtool
else
   echo "You run this script as root user."
   apt-get install wget git python-dev libblas-dev liblapack-dev gfortran \
   	gcc g++ libomp-dev libpthread-stubs0-dev make libtool
fi
\end{lstlisting}

These lines install the system dependencies, such as the compiler, the \python environment, which are required to compile and run \kratos.

\begin{lstlisting}[language=bash, basicstyle=\footnotesize, frame=single]
git clone https://github.com/vryy/kratos_bcn2.git
cd kratos_bcn2
KRATOS_ROOT_PATH=$(pwd)
\end{lstlisting}

This will check out, i.e. download and initialize, the \kratos repository located in \github. Additionally, a local system variable named \texttt{KRATOS\_ROOT\_PATH} is created for use later on during the configure process. This variable points to the location where the source code of \kratos is located.

\begin{lstlisting}[language=bash, basicstyle=\footnotesize, frame=single]
cd applications
git clone https://github.com/vryy/structural_application.git
git clone https://github.com/vryy/ExternalSolversApplication.git
\end{lstlisting}

These lines will check out the \texttt{structural\_application} and \texttt{ExternalSolversApplication} to the subfolder \texttt{applications} within the \kratos directory. These applications are sufficient to run a simple structural mechanics problem.

\begin{lstlisting}[language=bash, basicstyle=\footnotesize, frame=single]
cd external_libraries
echo "Get and compile Cmake."
wget http://www.cmake.org/files/v2.8/cmake-2.8.12.tar.gz
tar -xvf cmake-2.8.12.tar.gz
mv cmake-2.8.12 cmake-2.8.12_src
cd cmake-2.8.12_src
./configure --prefix=$(pwd)/../cmake-2.8.12
make install -j$(nproc)
\end{lstlisting}

\cmake is necessary to configure \kratos. \cmake is a build system which produce a \texttt{Makefile}. This file is then used to invoke the \texttt{make} command to compile the source code.

\begin{lstlisting}[language=bash, basicstyle=\footnotesize, frame=single]
wget http://sourceforge.net/projects/boost/files/boost/1.77.0/\
		boost_1_77_0.tar.bz2/download
mv download boost_1_77_0.tar.bz2
tar -xvf boost_1_77_0.tar.bz2
mv boost_1_77_0 boost_1_77_0_src
cd boost_1_77_0_src
./bootstrap.sh --prefix=$(pwd)/../boost_1_77_0
./b2 install -j$(nproc)
\end{lstlisting}

\boost library is used in \kratos for smart pointer and dense linear algebra. It is necessary to compile \boost. The version number \texttt{77} can be replaced by a newer version, however it is suggested to stick to the version provided in the installation script.

\section{Object-oriented programming}
% this section provides a rough overview of OOP, e.g. polymophism, inheritance. It does not mean to provide a detail discussion on the topic

\section{FEM computational workflow}
% typical computation flow of a FEM program, starting from the element loop to iteration through quadrature points and calling the material calculation routines.

\section{Structure of \kratos}
% structure of kratos that adheres to the FEM simulation flow in the previous section

%%%%%%%%%%%%%%%%%%%%%%%%%%%%%%%%%%%%%%%%%%%%%%%%%%%%%%%%%%%%%%%%%%%%%%%%%%%%%%%
% BIBLIOGRAPHY
%%%%%%%%%%%%%%%%%%%%%%%%%%%%%%%%%%%%%%%%%%%%%%%%%%%%%%%%%%%%%%%%%%%%%%%%%%%%%%%
% \newpage
% \bibliographystyle{sty/chicago_hurga}
% \bibliography{lit}

%%%%%%%%%%%%%%%%%%%%%%%%%%%%%%%%%%%%%%%%%%%%%%%%%%%%%%%%%%%%%%%%%%%%%%%%%%%%%%%
% END OF BIBLIOGRAPHY
%%%%%%%%%%%%%%%%%%%%%%%%%%%%%%%%%%%%%%%%%%%%%%%%%%%%%%%%%%%%%%%%%%%%%%%%%%%%%%

\end{document}

